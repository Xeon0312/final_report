% Options for packages loaded elsewhere
\PassOptionsToPackage{unicode}{hyperref}
\PassOptionsToPackage{hyphens}{url}
%
\documentclass[
  12pt,
]{article}
\usepackage{lmodern}
\usepackage{amssymb,amsmath}
\usepackage{ifxetex,ifluatex}
\ifnum 0\ifxetex 1\fi\ifluatex 1\fi=0 % if pdftex
  \usepackage[T1]{fontenc}
  \usepackage[utf8]{inputenc}
  \usepackage{textcomp} % provide euro and other symbols
\else % if luatex or xetex
  \usepackage{unicode-math}
  \defaultfontfeatures{Scale=MatchLowercase}
  \defaultfontfeatures[\rmfamily]{Ligatures=TeX,Scale=1}
  \setsansfont[]{Calibri Light}
\fi
% Use upquote if available, for straight quotes in verbatim environments
\IfFileExists{upquote.sty}{\usepackage{upquote}}{}
\IfFileExists{microtype.sty}{% use microtype if available
  \usepackage[]{microtype}
  \UseMicrotypeSet[protrusion]{basicmath} % disable protrusion for tt fonts
}{}
\makeatletter
\@ifundefined{KOMAClassName}{% if non-KOMA class
  \IfFileExists{parskip.sty}{%
    \usepackage{parskip}
  }{% else
    \setlength{\parindent}{0pt}
    \setlength{\parskip}{6pt plus 2pt minus 1pt}}
}{% if KOMA class
  \KOMAoptions{parskip=half}}
\makeatother
\usepackage{xcolor}
\IfFileExists{xurl.sty}{\usepackage{xurl}}{} % add URL line breaks if available
\IfFileExists{bookmark.sty}{\usepackage{bookmark}}{\usepackage{hyperref}}
\hypersetup{
  pdftitle={How the 2019 Canadian Federal Election would have been different if `everyone' had voted},
  pdfauthor={Cao Boyu},
  hidelinks,
  pdfcreator={LaTeX via pandoc}}
\urlstyle{same} % disable monospaced font for URLs
\usepackage[margin=1in]{geometry}
\usepackage{graphicx,grffile}
\makeatletter
\def\maxwidth{\ifdim\Gin@nat@width>\linewidth\linewidth\else\Gin@nat@width\fi}
\def\maxheight{\ifdim\Gin@nat@height>\textheight\textheight\else\Gin@nat@height\fi}
\makeatother
% Scale images if necessary, so that they will not overflow the page
% margins by default, and it is still possible to overwrite the defaults
% using explicit options in \includegraphics[width, height, ...]{}
\setkeys{Gin}{width=\maxwidth,height=\maxheight,keepaspectratio}
% Set default figure placement to htbp
\makeatletter
\def\fps@figure{htbp}
\makeatother
\setlength{\emergencystretch}{3em} % prevent overfull lines
\providecommand{\tightlist}{%
  \setlength{\itemsep}{0pt}\setlength{\parskip}{0pt}}
\setcounter{secnumdepth}{-\maxdimen} % remove section numbering
\usepackage{float}
\floatplacement{figure}{H}

\title{How the 2019 Canadian Federal Election would have been different if
`everyone' had voted\thanks{Code and data are available at:
\url{https://github.com/Xeon0312/final_report/}}}
\author{Cao Boyu}
\date{22 December 2020}

\begin{document}
\maketitle
\begin{abstract}
In this paper, we develop two models with the purpose of forecasting how
the 2019 Canadian Federal Election would have been different if
`everyone' had voted. Multilevel regression with postratification (MRP)
and Stacked regression with postratification models where constructed
using the 2019 Canadian Election Study survey and postsratified using
the 2017 General Social Survey. We concluded that if ``everyone'' votes,
there is a high probability that there is no obvious difference.

\textbf{Keywords:} Forecasting; 2019 Canadian Federal Election;
Multilevel Regression with Poststratification; No different
\end{abstract}

\hypertarget{introduction}{%
\section{1 Introduction}\label{introduction}}

The 2019 Canadian federal election (formally the 43rd Canadian general
election) was held on October 21, 2019, to elect members of the House of
Commons to the 43rd Canadian Parliament. The Liberal Party, led by
incumbent Prime Minister Justin Trudeau, won 157 seats to form a
minority government and lost the majority they had won in the 2015
election. The Liberals lost the popular vote to the Conservatives, which
marks only the second time in Canadian history that a governing party
formed a government while receiving less than 35 per cent of the
national popular vote.The Conservative Party, led by Andrew Scheer, won
121 seats and remained the Official Opposition. The Bloc Québécois,
under Yves-François Blanchet, won 32 seats to regain official party
status and became the third party for the first time since 2008. The New
Democratic Party, led by Jagmeet Singh, won 24 seats, its worst result
since 2004. The Green Party, led by Elizabeth May, saw its best election
results with three seats and for the first time received over one
million votes.

However, the gap between the Conservative Party and the Liberal Party is
not very large. When voters vote, they may be affected by other
conditions that may cause some voters to absent or abstain. The final
result may still be somewhat different from the real public opinion.
This time, we use multilevel regression with postratification (MRP) and
Stacked regression with postratification models to simulate the answer
if `everyone' had voted. For the data part, we are going to use 2019
Canadian Election Study survey and postsratified using the 2016 Canadian
Education Highlight census dataset. Details will be in the Model
section.

After a series of simulation experiments, we found that although the
proportion of votes among the parties has changed, the overall result is
the same as before. The Liberal Party is still elected by a narrow
margin. This reflects that the public opinion of Canadian voters has
been truly reflected and has not been affected by other factors.

\hypertarget{data}{%
\section{2 Data}\label{data}}

\hypertarget{ces}{%
\subsection{2019 CES}\label{ces}}

The 2019 Canadian Election Study was conducted to gather the attitudes
and opinions of Canadians during and after the 2019 federal election. It
continues the tradition of Canadian Election Studies started in 1965.
There are 2 data files for the 2019 CES - online survey and phone
survey. We only used online survey this time. The primary mandate of the
Canadian Election Study is to provide a thorough account of the
election, to underline the main reasons why people vote the way they do,
to indicate what does and does not change during the campaign and from
one election to another, and to highlight similarities Campaign period
survey start from 2019-09-13 to 2019-10-21. The population is all people
did the web sruvey during the period.

\begin{figure}
\includegraphics{report_files/figure-latex/makeGraph1-1} \caption{Party Voted in CES2019 survey}\label{fig:makeGraph1}
\end{figure}

As can be seen in Figure 1, The Liberal Party and the Conservative Party
are the two most mainstream parties with support ratings of 34\% and
33\% respectively. NDP and Green Party ranked third and fourth with 16\%
and 9\% respectively

\begin{figure}
\includegraphics{report_files/figure-latex/makeGraph2-1} \caption{How Respondents Plan to Vote in 2019 By Gender}\label{fig:makeGraph2}
\end{figure}

It can be seen from the Figure 2 that the proportion of men and women in
the voting population is similar, with slightly more women. Women
accounted for the majority of votes obtained by NDP.

\begin{figure}
\includegraphics{report_files/figure-latex/makeGraph3-1} \caption{How Respondents Plan to Vote in 2019 By Province}\label{fig:makeGraph3}
\end{figure}

It can be concluded from the Figure 3 that the proportion of people in
Ontario who voted is higher. Interestingly, the votes of the Bloc
Quebecois party only come from Quebec.

\begin{figure}
\includegraphics{report_files/figure-latex/makeGraph4-1} \caption{How Respondents Plan to Vote in 2019 By education}\label{fig:makeGraph4}
\end{figure}

The data obtained from the Figure 4 shows that most of the voting
population has higher education than high school. Those who voted for
the Conservative Party had an average education level higher than those
who voted for the Liberal Party.

\hypertarget{gss-data}{%
\subsection{GSS Data}\label{gss-data}}

The dataset we used for our analysis came from responses to the ``2017
General Social Survey (GSS): Families Cycle 31''. The 2017 GSS,
conducted from February 2nd to November 30th, 2017, is a sample survey
with a cross-sectional design. The target population includes all
non-institutionalized persons 15 years of age and older, living in the
10 provinces of Canada. The survey uses a new frame, created in 2013,
that combines telephone numbers (landline and cellular) with Statistics
Canada's Address Register, and collects data via telephone. Data are
subject to both sampling and non-sampling errors. Each record in the
survey frame was assigned to a stratum within its province. A simple
random sample without replacement of records was next performed in each
stratum. The target sample size (i.e.~the desired number of respondents)
for the 2017 GSS was 20,000 while the actual number of respondents was
20,602. The survey frame was created using two different components: 1.
Lists of telephone numbers in use (both landline and cellular) available
to Statistics Canada from various sources (telephone companies, Census
of population, etc.); 2. The Address Register (AR): List of all
dwellings within the ten provinces. The overall response rate for the
2017 GSS was 52.4\%.

The target population for the 2017 GSS included all persons 15 years of
age and older in Canada, excluding: 1. Residents of the Yukon, Northwest
Territories, and Nunavut; and 2. Full-time residents of institutions.
That will cause some data missing.

\begin{figure}
\includegraphics{report_files/figure-latex/makeGraph5-1} \caption{GSS By Sex}\label{fig:makeGraph5}
\end{figure}

\begin{figure}
\includegraphics{report_files/figure-latex/makeGraph6-1} \caption{GSS By Age group}\label{fig:makeGraph6}
\end{figure}

\begin{figure}
\includegraphics{report_files/figure-latex/makeGraph7-1} \caption{GSS By Education}\label{fig:makeGraph7}
\end{figure}

\begin{figure}
\includegraphics{report_files/figure-latex/makeGraph8-1} \caption{GSS By Province}\label{fig:makeGraph8}
\end{figure}

\hypertarget{model}{%
\section{3 Model}\label{model}}

A Bayesian multilevel logistic model with post-stratification was used
to predict the plurality for each state in the 2019 Canadian Federal
Election. We used a logistic model as they can associate binary and
continuous independent variables with a binary categorical response
variable. Unlike the previous assignment, the choice was made between
two parties. However, what we have to do in this homework is to make a
distribution forecast among the six parties. So this time I made 6
models to predict whether they will vote for the party. Since the
decision of whether they will vote for the party given is condensed to a
binary response variable, the logistic model is well suited for our goal
of predicting the winner. Linear models were unsuited for our goals as
they require the response variable to be a continuous variable such as
height. Our response variable was either 0 (representing not to vote for
this party) or 1( representing a vote for this party), which is binary
in nature. Through the logical model, we can express the probability
``p'' that an individual will vote for the selected party as a
logarithmic function of various dependent variables.
\[log(\frac{p}{1 - p}) = \beta_0 + \beta_1x_1 + \beta_2x_2 + ... + \beta_kx_k\]

In our case, we will consider the following variables for the model:
gender, age, education level, and province. We use them because they can
match in our two data. We decided to make the model multi-layered in
order to group voters by province. Since voting preferences and
demographic groups vary greatly in each province, we think it is
appropriate to analyze multiple states separately. The most important
thing is that there are three provinces in our GSS data that are
missing, and we need to predict separately by province.
\[P(\beta_0, \beta_1, …, \beta_k| y, X_1,...,X_k) \propto P(y, X_1,...,X_k|\beta_0, \beta_1, …, \beta_k) * P(\beta_0, \beta_1, …, \beta_k)\]
We used ``brm'' to fit Bayesian generalized multivariate multilevel
models. Then we used ``add\_predicted\_draws'' to add draws from the
posterior fit. We performed 4000 simulations for each person in the GSS
according to the model of the corresponding party. In each simulation, 0
or 1 was obtained, which respectively represented ``will vote for the
corresponding party'' and ``will not vote for the corresponding party''.
Average the simulation results to get the probability that this person
will vote for the corresponding party.
\[E({\rm votes}) \approx \frac{{\sum} \, {\rm draws}\,{*}\,{\rm vote}} {{\rm Number} \, {\rm of} \, {\rm draw}}.\]

\hypertarget{results}{%
\section{4 Results}\label{results}}

\begin{figure}
\includegraphics{report_files/figure-latex/makeGraph9-1} \caption{Result of Liberal Patry}\label{fig:makeGraph9}
\end{figure}

\begin{figure}
\includegraphics{report_files/figure-latex/makeGraph10-1} \caption{Result of Conservative Patry}\label{fig:makeGraph10}
\end{figure}

\begin{figure}
\includegraphics{report_files/figure-latex/makeGraph11-1} \caption{Result of Green Patry}\label{fig:makeGraph11}
\end{figure}

\begin{figure}
\includegraphics{report_files/figure-latex/makeGraph12-1} \caption{Result of ndp Patry}\label{fig:makeGraph12}
\end{figure}

\begin{figure}
\includegraphics{report_files/figure-latex/makeGraph13-1} \caption{Result of People's Patry}\label{fig:makeGraph13}
\end{figure}

\begin{figure}
\includegraphics{report_files/figure-latex/makeGraph14-1} \caption{Result of Bloc Québécois}\label{fig:makeGraph14}
\end{figure}

\begin{figure}
\includegraphics{report_files/figure-latex/makeGraph15-1} \caption{Result of Green Patry}\label{fig:makeGraph15}
\end{figure}

As can be seen from Figure 9 to Figure 15.

Alberta voted 1095.92 for the Conservative Party, far more than the
number of votes Alberta voted for other parties. Voting in British
Columbia is more scattered, with the most people voting for the
Conservative Party with 810.38 votes, followed by the Liberal Party with
744.26 votes. Manitoba and Saskatchewan have the highest approval
ratings for the Conservatives, which won 514.38 and 333.51 respectively.

New Brunswick, Newfoundland, Nova Scotia, Ontario, and Prince Edward
Island have the most support for the Liberal Party. Among them, Ontario
contributed 2095.5535 votes to the Liberal Party, Nova Scotia was
615.53575, Newfoundland was 503.899, New Brunswick was 483.23125, and
Prince Edward Island voted 283.5632.

It is worth mentioning that the Quebec Party, all of their votes are
from Quebec, which is 1015.7825. However, the Liberal Party received
1,292.01225 votes, making it the most supported party in Quebec by a
narrow margin.

\begin{figure}
\includegraphics{report_files/figure-latex/makeGraph16-1} \caption{Over all predict}\label{fig:makeGraph16}
\end{figure}

\hypertarget{discussion}{%
\section{5 Discussion}\label{discussion}}

\pagebreak

\hypertarget{appendix}{%
\section{Appendix}\label{appendix}}

\pagebreak

\hypertarget{references}{%
\section{References}\label{references}}

Bürkner, P (2017). brms: An R Package for Bayesian Multilevel Models
Using Stan. Journal of Statistical Software, 80(1), 1-28.
\url{doi:10.18637/jss.v080.i01}

Bürkner, P (2018). Advanced Bayesian Multilevel Modeling with the R
Package brms. The R Journal, 10(1), 395-411.
\url{doi:10.32614/RJ-2018-017}

Gelman, A., \& Kennedy, L. (2020). ``Know your population and know your
model: Using model-based regression and post-stratification to
generalize findings beyond the observed sample.'' Retrieved from:
\url{https://arxiv.org/pdf/1906.11323.pdf}

JJ Allaire et al.~(2020). rmarkdown: Dynamic Documents for R. R package
version 2.3. Retrieved from: \url{https://rmarkdown.rstudio.com}.

R Core Team. (2020). R: A language and environment for statistical
computing. R Foundation for Statistical Computing, Vienna, Austria.
Retrieved from: \url{https://www.R-project.org/}.

Robinson, D., Hayes, A., \& Couch, S. (2020). broom: Convert Statistical
Objects into Tidy Tibbles. R Package version 0.7.0 Retrieved from:
\url{https://CRAN.R-project.org/package=broom}.

Rudis, B. (2020). statebins: Create United States Uniform Cartogram
Heatmaps. R package version 1.4.0. Retrieved from:
\url{https://CRAN.R-project.org/package=statebins}.

Stephenson, Laura B; Harell, Allison; Rubenson, Daniel; Loewen, Peter
John, (2020) `2019 Canadian Election Study - Online Survey',
\url{https://doi.org/10.7910/DVN/DUS88V}, Harvard Dataverse, V1

Wickham, H. (2016). ggplot2: Elegant Graphics for Data Analysis.
Springer-Verlag. New York.

Wickham, H., \& Miller, E. (2020). haven: Import and Export `SPSS',
`Stata' and `SAS' Files. R package version 2.3.1. Retrieved from:
\url{https://CRAN.R-project.org/package=haven}.

Wickham et al.~(2019). Welcome to the tidyverse. Journal of Open Source
Software, 4(43), 1686. Retrieved from:
\url{https://doi.org/10.21105/joss.01686}.

Wu, C., \& Thompson, M. (2020). Sampling Theory and Practice, Springer.

\end{document}
